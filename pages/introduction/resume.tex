\section{Introduction}

Dans le cadre de ma formation développeur web et web mobile, j’ai eu la chance d’assister Mr. Vincent VAURETTE qui est en charge de la cellule de transition numérique de la Fédération Française de Volley-Ball (FFVB).\\
Le but de son travail est de développer des outils qui facilitent la gestion de tout ce qui est en lien avec le monde du volley-ball, et de redonner un coup de jeune à l’intranet de la fédération.\\
Durant trois mois, j’ai eu la responsabilité de la transition numérique de l’interface de désignation des arbitres.

\vspace{1cm}

\includegraphics{logo.jpg}

La Fédération Française de Volley-Ball est une association créée en 1936 qui constitue l’instance dirigeante du volley-ball en France. 
Plusieurs centaines de matchs se déroulent par week-end, et la désignation des arbitres les concernant est entrêmement chronophage.


\begin{commentaire}
    Le site actuel de la fédération date de 2009 et manque d’ergonomie. Celui-ci demande plusieurs click pour effectuer une seule désignation et ne permet pas d’afficher toutes les informations nécessaires pour celles-ci. Les gestionnaires sont alors contraints de passer par un tableur externe pour procéder aux désignations, puis ensuite de rentrer manuellement celles-ci pour chaque match.
\end{commentaire}


L’objectif principal qui m’a été confié a été la création d’un outil permettant d’automatiser ces désignations. Celui-ci devait pouvoir associer un ou plusieurs arbitres habilités et disponibles à chaque match d’une période donnée, tout en laissant le choix de leur modification.