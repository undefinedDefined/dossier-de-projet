%----------------------------------------------------------------
% Packages
%----------------------------------------------------------------

\usepackage{natbib} % Pour pouvoir utiliser une bibliographie externe
\usepackage[french]{babel}	% Pour préciser la langue du document
\usepackage[utf8]{inputenc}	% Précise comment le texte est saisi : cela permet de tapper directement les accents
\usepackage[T1]{fontenc}	% Précise la façon dont le document actuel est encodé
\usepackage{setspace}
\usepackage[margin=2.5cm]{geometry} % Précise les marges du document

% Boxes quotation
\usepackage{tcolorbox}

% icones
\usepackage{fontawesome5}

%Sections
%----------------------------------------------------------------
%\usepackage{newclude} % Pour pouvoir utiliser l'étoile après \inculde pour éviter les sauts de page. Ce package a des problême de compatibilité avec la package natbib
%\renewcommand\thesection{} % Pour éviter la numérotation des sections
%----------------------------------------------------------------

%Autres packages et commandes utiles
%----------------------------------------------------------------
\usepackage{amsmath,amsthm,amssymb,amsfonts}	% Pour pouvoir inclure certains symboles et environnements mathématiques
\usepackage{enumerate} % Pour mieux gérer la commande enumerate dans les sections
\usepackage{graphicx}	% Pour inclure des images
\graphicspath{{assets/}}
\usepackage{color}	% Pour inclure du texte en couleur
\usepackage{units}	% Pour pouvoir tapper les unités correctement
\usepackage{pgf,tikz}	% Utilisation du module tikz, qui permet de tracer des belles images
% \usetikzlibrary{arrows} % Quand on exporte une image GeoGebra, on a besoin de préciser cela
\usepackage{hyperref}	% Pour include des liens dans le document
% \usepackage{cprotect}	% Pour pouvoir personaliser la légende des figures
%----------------------------------------------------------------

